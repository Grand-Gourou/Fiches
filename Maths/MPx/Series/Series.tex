\documentclass[french, a4paper, 11pt, twocolumn]{article}

\usepackage[utf8]{inputenc} % ~ Encodage
\usepackage[T1]{fontenc}    % ~ Encodage
\usepackage[left=1cm, right=1cm]{geometry} % ~ Mise en page et marges
\usepackage{amssymb} % ~ Pour écrire les maths
\usepackage{xspace}  % ~ Commandes à texte
\usepackage{varioref} % ~ Références croisées
\usepackage{enumitem} % ~ Listes
\usepackage{xcolor}   % ~ Couleurs fs
\usepackage{float}
\usepackage{tikz}
\usepackage[straightvoltages]{circuitikz}
\usepackage[squaren, cdot, derived]{SIunits}
\usepackage{graphicx}
\usepackage[f]{esvect}
\usepackage[many]{tcolorbox}
\usepackage{euler}
\usepackage[nointegrals]{wasysym}
\usepackage[french]{babel}


%______FONCTIONS______%
\newcommand{\ssi}{si et seulement si\xspace}		% ~ ssi
\newcommand{\inv}[1]{\dfrac{1}{#1}}
% --ENSEMBLES--%
\newcommand{\N}{\mathbb{N}}   % ~ Entiers naturels
\newcommand{\Z}{\mathbb{Z}}   % ~ Entiers relatifs
\newcommand{\D}{\mathbb{D}}   % ~ Decimaux
\newcommand{\Q}{\mathbb{Q}}   % ~ Rationnels
\newcommand{\R}{\mathbb{R}}   % ~ Réels
\newcommand{\C}{\mathbb{C}}   % ~ Complexes
% --TRIGO--%
\let\cosh\relax
\DeclareMathOperator{\cosh}{ch}       % ~ cosinus hyperbolique
\DeclareMathOperator{\sh}{sh}         % ~ sinus hyperbolique
\let\tanh\relax
\DeclareMathOperator{\tanh}{th}       % ~ tangente hyperbolique
\DeclareMathOperator{\argch}{Argch}   % ~ Argument cosinus hyperbolique
\DeclareMathOperator{\argsh}{Argsh}   % ~ Argument sinus hyperbolique
\DeclareMathOperator{\argth}{Argth}   % ~ Argument tangente hyperbolique
\DeclareMathOperator{\cotan}{cotan}   % ~ cotangente
% --PARENTHESES--%
\newcommand{\po}{\left(}         % ~ (
\newcommand{\pf}{\right)}        % ~ )
\newcommand{\pof}[1]{\po #1 \pf} % ~ ( )
\newcommand{\co}{\left[}         % ~ [
\newcommand{\cf}{\right]}        % ~ ]
\newcommand{\cof}[1]{\co #1 \cf} % ~ [ ]
\newcommand{\chof}[1]{\left\langle #1 \right\rangle } % ~ < >
\newcommand{\interoo}[2]{\left]#1\,;#2\right[}   % ~ ]a,b[
\newcommand{\interof}[2]{\left]#1\,;#2\right]}   % ~ ]a,b]
\newcommand{\interfo}[2]{\left[#1\,;#2\right[}   % ~ [a,b[
\newcommand{\interff}[2]{\left[#1\,;#2\right]}   % ~ [a,b]
% --VECTEURS--%
\newcommand{\ux}{\vect{u_x}}          % ~ Vecteur ux
\newcommand{\uy}{\vect{u_y}}          % ~ Vecteur uy
\newcommand{\uz}{\vect{u_z}}          % ~ Vecteur uz
\newcommand{\ur}{\vect{u_r}}          % ~ Vecteur ur
\newcommand{\uth}{\vect{u_\theta}}    % ~ Vecteur utheta
\newcommand{\uph}{\vect{u_\varphi}}   % ~ Vecteur uphi
\newcommand{\om}{\vect{OM}}           % ~ Vecteur position
\newcommand{\vvi}{\vect{v}}           % ~ Vecteur vitesse
\newcommand{\vvio}{\vect{v_0}}        % ~ Vecteur v0
\newcommand{\va}{\vect{a}}            % ~ Vecteur accélération
\newcommand{\vp}{\vect{p}}            % ~ Vecteur quantité de mouvement
\newcommand{\fr}{\vect{F_r}}          % ~ Vecteur force de rappel
\newcommand{\vabla}{\vect{\nabla}}    % ~ nabla
\newcommand{\grad}{\vect{\mathrm{grad}}}  % ~ grad
\DeclareMathOperator{\diverg}{div}        % ~ div
\newcommand{\rot}{\vect{\mathrm{rot}}}    % ~ rot

\newtcolorbox{theoreme}[2][]
{
  enhanced,
  attach boxed title to top left={yshift=-3.4mm, xshift = -2.3mm},
  adjusted title=#2,
  colback=white, colframe=black,
  colbacktitle=white, coltitle=black, fonttitle=\bfseries,
  breakable, sharp corners,
  boxed title style={colback=white, sharp corners, colframe=white},
  boxrule = 0.5mm, drop fuzzy shadow
}

\newtcolorbox{definition}[1][]
{
  enhanced,
  attach boxed title to top left={yshift=-3.4mm, xshift = -2.3mm},
  adjusted title=Définition,
  colback=white, colframe=black,
  colbacktitle=white, coltitle=black, fonttitle=\bfseries,
  breakable, sharp corners,
  boxed title style={colback=white, sharp corners, colframe=white},
  boxrule = 0.5mm, drop fuzzy shadow
}


\newcommand{\ooint}{\ocircle\hspace{-3.65mm}\int\hspace{-2mm}\int}

\title{Séries}
\author{Martin \textsc{Andrieux}}
\date{}

\begin{document}
\maketitle

\begin{definition}
  Une série est dite \emph{convergente} quand la suite des sommes partielles converge. Dans le cas contraire, elle est dite \emph{divergente}.
\end{definition}

\section{Séries à termes positifs}
\begin{theoreme}{Théorèmes de comparaison}
  \begin{itemize}[label=$\bullet$]
    \item Si pour $n>n_{2}$ on a $x_{n}<y_{n}$, alors
      \[\begin{cases}
          \sum y_{n} \text{ cv} \implies \sum x_{n} \text{ cv}\\
          \sum x_{n} \text{ div} \implies \sum y_{n} \text{ div}\\
      \end{cases}\]
    \item Même conclusion si $x_{n} \underset{+\infty}{=} \mathcal{O}(y_{n})$
    \item Si $x_{n} \underset{+\infty}{\sim} y_{n}$, les séries sont de même nature. De plus, si les séries divergent, leur sommes partielles sont équivalentes, sinon, leur restes sont équivalents.
  \end{itemize}
\end{theoreme}

\begin{theoreme}{Séries de référence}
  \begin{itemize}[label=$\bullet$]
    \item $\sum q^{n}$ converge \ssi $q<1$. On a alors
      \[\sum_{n=0}^{+\infty}q^{n} = \inv{1-q}\]
    \item $\sum \frac{1}{n^{\alpha}}$ converge \ssi$\alpha > 1$.
  \end{itemize}
\end{theoreme}

\begin{theoreme}{Critère de d'Alembert}
  Avec $\sum x_{n}$ une série à termes positifs,
  Si $\frac{x_{n+1}}{x_{n}}$ a une limite $l$, alors
  \begin{itemize}[label=$\bullet$]
    \item Si $l<1$, la série converge
    \item Si $l>1$, la série diverge
  \end{itemize}
\end{theoreme}

\section{Séries quelconques}
\begin{definition}
  Avec $\sum x_{n}$ une série à termes  réels, si $\sum\left| x_{n}\right|$ converge, alors la série converge. On la qualifie alors de série \emph{absolument convergente}. Si la série converge sans converger absolument, elle est \emph{semi-convergente}.
\end{definition}

\begin{theoreme}{Théorème spécial des séries alternées}
  Avec $\sum_{n\geqslant 0} \alpha_{n}$ et $\forall n\, \alpha_{n}\geqslant 0$,
  si $\alpha_{n}$ décroit vers $0$, alors la série est convergente.
\end{theoreme}

\begin{theoreme}{Utilisation de DL}
  Avec $\sum u_{n}$, si tous les termes du développement limité de $u_{n}$ convergent, alors la série converge.
\end{theoreme}

\begin{theoreme}{Transformation d'Abel}
  Réécrire le terme général de la série comme différence de sommes partielles bornées permet d'établir la nature de la série.
  \tcblower
  Le théorème d'Abel dit que si $b_{n}$ décroit vers $0$ et si \[\ \pof{A_{n}}_{n\geqslant 0}=\sum_{k=0}^n a_k\] est bornée, alors $\sum a_{n}b_{n}$ converge.
\end{theoreme}

\begin{theoreme}{Produit de Cauchy}
  Soient $\sum a_{n}$ et $\sum b_{n}$ deux séries à termes réels. Le produit de Cauchy de ces deux séries est:
  \[\sum_{n\geqslant 0}\sum_{i=0}^{n}a_{i}b_{n-i}\]
  Si les séries des $a_{n}$ et des $b_{n}$ sont absolument convergentes, alors leur produit de Cauchy est absolument convergent et tend vers le produit des limites des deux séries
\end{theoreme}

\end{document}
