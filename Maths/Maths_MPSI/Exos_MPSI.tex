\documentclass[a4paper, 11pt, french]{article}


\usepackage[utf8]{inputenc} 
\usepackage[T1]{fontenc}
\usepackage{geometry}
\usepackage{enumitem}
\usepackage{amssymb}
\usepackage{mathtools}
\usepackage{amsmath}
\usepackage{amsfonts}
\usepackage{varioref}
\usepackage{graphicx}
\usepackage{gensymb}
\usepackage{xcolor}
\usepackage{systeme}

\newcommand{\R}{\mathbb{R}}
\newcommand{\C}{\mathbb{C}}
\newcommand{\N}{\mathbb{N}}
\newcommand{\Z}{\mathbb{Z}}
\newcommand{\K}{\mathbb{K}}
\newcommand{\Q}{\mathbb{Q}}
\newcommand{\p}{\wedge}
\newcommand{\D}{\mathcal{D}}
\newcommand{\dx}{\mathrm{d}x}
\newcommand{\dt}{\mathrm{d}t}
\newcommand{\tr}{\mathrm{tr}}
\newcommand{\ev}{espace vectoriel}
\newcommand{\eve}{espace vectoriel euclidien}
\newcommand{\sce}{système complet d'évènements}
\newcommand{\al}{\alpha}
\newcommand{\be}{\beta}
\newcommand{\De}{\Delta}
\newcommand{\de}{\delta}
\newcommand{\la}{\lambda}
\newcommand{\te}{\theta}
\newcommand{\si}{\sigma}
\newcommand{\Om}{\Omega}
\newcommand{\om}{\omega}
\newcommand{\ph}{\varphi}
\newcommand{\ep}{\varepsilon}
\newcommand{\som}[2]{\overset{#2}{\underset{#1}{\sum}}}
\newcommand{\produit}[2]{\overset{#2}{\underset{#1}{\prod}}}
\newcommand{\thm}{\textcolor{red}{\underline{Théorème} }}
\newcommand{\ppt}{\textcolor{red}{\underline{Propriété :} }}
\newcommand{\limit}[1]{\underset{#1}{\rightarrow}}
\newcommand{\eq}[1]{\underset{#1}{\sim}}
\newcommand{\inv}[1]{\frac{1}{#1}}
\newcommand{\acc}[1]{\left\{ #1 \right\}}
\newcommand{\mat}[2]{\left( \begin{array}{#1} #2 \end{array} \right)}
\newcommand{\coeff}[2]{\dfrac{f\left(#1\right) - f\left(#2\right)}{#1-#2}}



\title{Maths MPSI}
\author{Maillet Nathan\\MP*}
\date{}

\begin{document}
	\maketitle
	\section*{Equations différentielles}
	\begin{itemize}
 		\item Résoudre $x^2y'+(2x-1)y=0$ sur chacun des intervalles $\R^{+*}$ et $\R^{-*}$. Cette équation a-t-elle des solutions sur $\R$ ? Si oui, les 					préciser.
	\end{itemize}	

	\section*{Continuité}
	\begin{itemize}
 		\item Résoudre $f:\R\rightarrow \R$ continue / $\forall(x,y), f(x+y)=f(x)f(y)$
 		\item $f$ continue sur $\R$ / $\underset{+\infty}\lim f=\underset{-\infty}\lim f=+\infty \implies f$ admet un minimum sur $\R$
 		\item  $f \in C^1([a,b]),$ justifier l'existence de $M_1= \sup_{[a,b]}(|f'|)$
	\end{itemize}	

	\section*{Suites}
	\begin{itemize}
 		\item Soit $(u_n)_{n \in \N}$ / $u_{n+1}=1+\inv{1+u_n}.$ Justifier que $|u_{n+1}-\sqrt{2}|\leqslant \frac{\sqrt{2}-1}{2}|u_n-\sqrt{2}|$
 		\item $b>a>0, a_0=a,b_0=b, a_{n+1}=\inv{2}(a_n+b_n),b_{n+1}=\sqrt{a_{n+1}b_n}.$ Calculer la limite de $a_n$ et $b_n$
 		\item $f(t)=ln(1+t), u_0>0, u_{n+1}=f(u_n)$ Limite de $u_n$ ?
	\end{itemize}
	 

	\section*{Propiétés de $\R$ en tout genre}
	\begin{itemize}
 		\item Montrer la densité de $\Q$ dans $\R$
	\end{itemize}
	 

	\section*{Dérivabilité}
	\begin{itemize}
 		\item Taylor-Lagrange ordre 2 : Soit $f \in C^1[,b], f \in \D^2]a,b[.$ Montrer qu'il existe ${c \in ]a,b[} / f(b)=f(a)+(b-a)f'(a)+\frac{(b-a)^2}{2}f"(c)$
		\item Dérivée n-ième de $e^x\cos(x)$
		\item $f(x)=\inv{2+x}, u_0=1, u_{n+1}=f(u_n).$ Limite de $u_n$ ?
	\end{itemize}

	\section*{Lois de compositions internes}
	\begin{itemize}
 		\item Soit $\mathcal{H}=\acc{\left( \begin{array}{c c} a & -\bar{b} \\ b & \bar{a} \end{array} \right), (a,b) \in \C^2}$ Quelle structure a-t-on ? 
	\end{itemize}

	\section*{Arithmétique}
	\begin{itemize}
 		\item Résoudre $\systeme{x\equiv 2 [7], x \equiv 1[8], x \equiv 3[9]}$
		\item Prouver que $374 935=401*17*11*5$ divise $3^{400}-1$ et donner le reste de la division euclidienne de $(100^{200})^{300}$ par $23$
		\item $a \p b =1,$ prouver que $(a+b)\p ab=1$
		\item Quelle condition est nécessaire et suffisante sur $n$ pour que $\Z/n\Z$ soit un corps ?
	\end{itemize}

	\section*{Polynômes}
	\begin{itemize}
 		\item Déterminer $P / (X+3)P(x)=XP(X+1)$ et $P / P(X^2)=P(X)P(X-1)$ sur $\C[X]$
 		\item Trouver toutes les valeurs de n pour lesquelles $X^2+X+1$ divise $(X+1)^n-X^n-1$
 		\item Décomposer en facteur irréductibles de $\R[X] X^8+X^4+1, \frac{x^3}{(x+1)^4(x+2)^2}$ et calculer $S_n=\som{k=1}{n}\frac{k}{k^4+k^3+1}$
	\end{itemize}

	\section*{Intégration}
	\begin{itemize}
 		\item Convergence, calcul et équivalent de $w_n= \int_0^{\frac{\pi}{2}}\sin^n(t)\dt$
 		\item Soit $f \in C^1, I_n=\int_a^b f(t)\sin(nt)\dt.$ Montrer que $\underset{\infty}\lim I_n=0$
 		\item Ensemble de définition et dérivée de $\int_x^{x^2}\frac{\dt}{ln(t)}$
 		\item $\underset{\infty}\lim \, n^3\som{k=1}{n}\inv{n^4+k^2n^2+k^4}$
	\end{itemize}

	\section*{Développements limités}
	\begin{itemize}
 		\item Décrire le graphe au voisinage de $x=1$ de $f(x)=\frac{\ln(3-2x)}{x}$
		\item Donner le comportement graphique en + et - $\infty$ de $g(x)=(x-3)e^{\inv{1+2x}}$
	\end{itemize}

	\section*{Espaces vectoriels}
	\begin{itemize}
 		\item Soit des réels $(\la_k)_1^n$ 2 à 2 distincts. Montrer que la famille de fonctions de $\R$ vers $\R$ définie par $x \mapsto e^{\la_kx}$ est une famille libre.
 		\item Soit $f \in L(E) / f^2-7f+12Id=\om.$ Montrer que $Ker(f-3Id) \oplus Ker(f-4Id)=E$
	\end{itemize}

	\section*{Applications linéaires}
	\begin{itemize}
 		\item Soit $(f,h) \in L(E,F)\times L(F,H)$ Montrer que $h\circ f=\om \iff Im(f) \subset Ker(h)$
 		\item $(u,v) \in L(E)$ prouver $(u\circ v=u, v\circ u=v) \iff (u,v$ projecteurs et ${Ker(u)=Ker(v))}$
 		\item Soit $E=C^0(\R^+,\R), T : E \rightarrow E / T(f)=F, F(0)=f(0), F(x)=\inv{x}\int_0^xf(t)\dt$ T est-il un endomorphisme ? Quels sont ses valeurs et vecteurs propres ?
		\item Soit $A=\acc{\ph \in L(E) / \ph=u\circ f\circ v, f\in L(E)} A :$ \ev \, ? $Ker(\ph), {Im(\ph) ?}\, (\dim(A) ?)$
	\end{itemize}

	\section*{Matrices}
	\begin{itemize}
 		\item Calculer $A^n$ dans chacun des cas : $A=\left( \begin{array}{c c c} 1 & 1 & 1 \\ 1 & 0 & 0 \\ 1 & 0 & 0 \end{array} \right)$ (rec), $A=\left( \begin{array}{c c c} 1 & 1 & 0 \\ 0 & 1 & 1 \\ 0 & 0 & 1 \end{array} \right)$ (binôme), $A=\left( \begin{array}{c c c} 0 & 4 & -4 \\ 1 & 3 & -1 \\ -3 & 3 & -1 \end{array} \right)$ (polynôme)
		\item $A=\left( \begin{array}{c c c} 1 & 0 & 0 \\ 0 & 0 & -1 \\ 0 & 1 & 2 \end{array} \right)$ est-elle diagonalisable ?
	\end{itemize}

	\section*{Déterminants}
	\begin{itemize}
 		\item Soit $M=(sup(i,j)),$ calculer $\det(M)$
 		\item Calcul de : $\begin{vmatrix} 
						  1 & 1 & 1 & \cdots & 1 \\
						  -1 & 2 & 0 & \cdots & 0 \\
						  0 & \ddots & \ddots & \ddots & \vdots \\
						  \vdots & \ddots & \ddots & \ddots & 0 \\
						  0 & \cdots & 0 & -1 & 2
					\end{vmatrix}$
 		\item Calcul du déterminant de Van der Monde : $\begin{vmatrix} 
						  1 & \al_1 & (\al_1)^2 & \cdots & (\al_1)^{n-1} \\
						  1 & \al_2 & (\al_2)^2 & \cdots & (\al_2)^{n-1} \\
						  \vdots & \vdots & \vdots & \vdots & \vdots \\
						  1 & \al_n & (\al_n)^2 & \cdots & (\al_n)^{n-1}
					\end{vmatrix}$
	\end{itemize}

	\section*{Systèmes linéaires et espaces affines}
	\begin{itemize}
 		\item Résoudre : \systeme{x+y+\la z=1, x+\la y+z=1, \la x+y+z=1}
		\item Décrire l'intersection des ensembles $\left\{\begin{aligned}
													      x&=-1+2\al\\   
													      y&=-\al\\
													      z&=3+4\al\\
   												\end{aligned} \right.$
			et $\left\{\begin{aligned}
						x&=1+\al-\be\\   
						y&=-2\al -3\\
						z&=-1+4\al-\be\\
   				\end{aligned} \right.$
	\end{itemize}

	\section*{Séries}
	\begin{itemize}
 		\item Nature des séries : $\som{k=1}{N}\inv{k^{0.8}\ln(k)}, \som{k=2}{N}\inv{k\ln^2(k)}$ et $\som{k=1}{N}\frac{\ln^5(k)}{k^{1.1}}$
 		\item Soit $G_n=\inv{(2n)!2^{2n}}\produit{k=1}{2n}(2k-1).$ Donner la nature de la série des $G_n$
 		\item Montrer que la suite définie par $u_n=1+\inv{2}+\cdots+\inv{n}-\ln(n)$ converge et déterminer la nature de la série $\som{n\geqslant 1}{}\inv{n}(1+\inv{2}+\cdots+\inv{n})^{-1}$
	\end{itemize}

	\section*{Espaces euclidiens}
	\begin{itemize}
 		\item Résoudre $\vec{x}+\vec{a}\p \vec{x}=\vec{b}$ dans $\R^3$
		\item Caractériser géométriquement $A=\inv{3} \left( \begin{array}{c c c} 2 & -2 & 1 \\ -2 & -1 & 2 \\ 1 & 2 & 2 \end{array} \right)$ et $B=\inv{3} \left( \begin{array}{c c c} -2 & -1 & 2 \\ 2 & -2 & 1 \\ 1 & 2 & 2 \end{array} \right)$
		\item Soit $G=\acc{\vec{x} \in R^4 / x_1+x_2=x_3+x_4=0}$ et $\vec{u}=(1,-1,0,1).$ Donner la matrices de $P_G$ et de $P_{G^{\perp}}.  \, d(\vec{u},G) = \, ?$
		\item $f : \R^2 \rightarrow \R, f(a,b)=\int_0^1(t^2-at-b)^2e^{-t}\dt$ minima sur $\R^2$ ?
	\end{itemize}

	\section*{Convexité}
	\begin{itemize}
		\item Montrer l'inégalité de Hölder : Soit $p,q>0 / \inv{p}+\inv{q}=1, (a_i),(b_i)>0 \implies {\som{}{}a_ib_i \leqslant (\som{}{}(a_i)^p)^{\inv{p}}(\som{}{}(b_i)^p)^{\inv{p}}}$
		\item Montrer l'inégalité de Minkowski : Soit $p,q>0 / \inv{p}+\inv{q}=1, (a_i),(b_i)>0 \implies {(\som{i=1}{n}(a_i+b_i)^p)^{\inv{p}}\leqslant (\som{}{}(a_i)^p)^{\inv{p}} + (\som{}{}(b_i)^p)^{\inv{p}}}$
 		\item Montrer que la convexité de $f$ est équivalente à :
	\[\forall\,(a,b,c)\, /\, a<b<c,\]
	\begin{equation}
		\coeff{b}{a} \leqslant \coeff{c}{a} \leqslant \coeff{c}{b}
		\label{ineq}
	\end{equation}
	\end{itemize}

	\section*{Probabilités}
	\begin{itemize}
		\item Combien y a-t-il de surjections d'un ensemble de cardinal $n+2$ vers un ensemble de cardinal $n$ ?
		\item $n$ personnes passent le permis avec chacun une probabilité $p$ de l'avoir. Les recalés repassent dans exactement les même conditions. Soient les variables aléatoires $X :$ obtention du permis après 1 passage et $Y :$ obtention du permis après 2 passages et $Z=X+Y. P(Z=l) =?$
		\item Soit un avion avec 400 places : en moyenne, 8\% des passagés ayant réservés sont absents. La compagnie enregistre 420 réservations. Majorer la probabilité qu'il n'y ait pas assez de place.
		\item Déterminer une probabilité sur $\Om =\acc{1,2,\cdots,n}$ telle que la probabilité de l'événement ${1,2,\cdots,k}$ soit proportionnelle à $k^2$.
		\item On considère $N$ coffres avec une probabilité $p$ un trésor a été placé dans l'un de ces coffres, chaque coffre pouvant être choisi de façon équiprobable. On a ouvert $N-1$ coffres sans rien trouver. Quelle est la probabilité pour qu'il figure dans le dernier coffre ?
		\item Soient $A_1,A_2,\cdots,A_n \, n$ évènements indépendants. Montrer que \\ ${P(A_1\cup A_2 \cup \cdots \cup A_n)=1-(1-P(A_1))\cdots(1-P(A_n))}.$ Une personne atteint le centre d'une cible au tire avec une probabilité de $0.04.$ Combien doit-elle faire d'essais pour l'atteindre avec une probabilité supérieure à $0.95$ ?
	\end{itemize}

\end{document}