\documentclass[french, a4paper, 11pt, twocolumn]{article}
\usepackage[left=1cm, right=1cm]{geometry}
\usepackage{marsu}
\usepackage{euler}
\usepackage[nointegrals]{wasysym}

\newtcolorbox{cadre}[2][]
{
  enhanced,
  attach boxed title to top left={yshift=-3.4mm, xshift = -2.3mm},
  adjusted title=#2,
  colback=white, colframe=black,
  colbacktitle=white, coltitle=black, fonttitle=\bfseries,
  breakable, sharp corners,
  boxed title style={colback=white, sharp corners, colframe=white},
  boxrule = 0.5mm, drop fuzzy shadow
}
\newcommand{\ooint}{\ocircle\hspace{-3.65mm}\int\hspace{-2mm}\int}

\title{Électromagnétisme}
\author{Martin \textsc{Andrieux}}
\date{}

\begin{document}
\maketitle

\section{Analyse vectorielle}
\begin{cadre}{Circulation}
  \[\mathcal C = \int_A^B \vv E(M)\cdot\vv{d M}\]
  \[\mathcal C = \oint \vv E(M)\cdot\vv{d M}\]
\end{cadre}

\begin{cadre}{Flux}
  \[\Phi = \ooint_{(S)} \vv E(M)\cdot\vv{d S}\]
  \[\Phi = \ooint_{(S)} \vv E(M)\cdot\vv{d S}\]
\end{cadre}

\begin{cadre}{Théorème de Gauss}
  Dans un champ électrique:
  \[\ooint_{(S)}\vv E(M)\cdot\vv{d S} = \dfrac{Q_{int}}{\varepsilon_{0}}\]
\end{cadre}

\begin{cadre}{Énergie}
  Pour $n$ charges ponctuelles:
  \[U_{e} = \dfrac{1}{2}\sum_{i=1}^n q_{i}v_{i}\]
  Pour une distributuion continue:
  \[U_{e} = \inv{2}\iiint_{\text{espace}}\varepsilon_{0}{\vv E}^{2}(M) d\tau\]
\end{cadre}

\begin{cadre}{Théorème d'Ostrogradski}
  \[\ooint_{(S)}\vv E \cdot \vv{d S} = \iiint_{(\mathcal V)} \diverg\po\vv E\pf d\tau\]
\end{cadre}

\begin{cadre}{Équation de Poisson}
  \[\Delta V + \dfrac{\rho}{\varepsilon_{0}} = 0\]
  En l'absence de charges:
  \[\Delta V = 0\]
\end{cadre}

\begin{cadre}{Théorème de Stokes}
  \[\oint \vv E\cdot\vv{d l} = \iint_{(S)}\rot \vv E \cdot \vv{d S}\]
\end{cadre}

\begin{cadre}{Équivalence}
  \[\vv E = -\grad V \iff \oint_{(C)}\vv E\cdot \vv{d l} = 0 \iff \rot\vv E = \vv0\]
\end{cadre}

\section{Dipôle électrostatique}
\begin{cadre}{Moment dipolaire}
  \[\vv p = q \vv{NP}\]
\end{cadre}

\begin{cadre}{Potentiel loin d'un dipôle}
  \[V(M) = \dfrac{p\cos(\theta)}{4\pi\varepsilon_{0}r^{2}}\]
  D'où:
  \[E_{r} = \dfrac{2p\cos(\theta)}{4\pi\varepsilon_{0}r^{3}}\text{ et } E_{\theta} = \dfrac{p\sin(\theta)}{4\pi\varepsilon_{0}r^{3}}\]
\end{cadre}

\begin{cadre}{Force exercée sur un dipôle}
  Force:
  \[\vv F = \pof{\vv p \cdot \grad}\vv E(M) = \po\vv p\cdot\vabla\pf\cdot\vv E\]
  Moment pour un champ variant peu:
  \[\vv\Gamma(M) = \vv p \wedge \vv E(M)\]
\end{cadre}

\begin{cadre}{Énergie Potentielle}
  \[E_{p} = -\vv p\cdot \vv E\]
\end{cadre}

\section{Conducteurs}
\begin{cadre}{Équilibre}
  \[\vv E = \vv 0 \text{ donc } V = 0\]
  Or, $\diverg \vv E = 0$ donc $\rho = 0$. La densité volumique de charge est nulle dans tout le volume du conducteur: la charge se localise uniquement sur la surface.
  D'où:
  \[\vv {E_{ext}} = \dfrac{\sigma}{\varepsilon_{0}}\vv n\]
\end{cadre}

\begin{figure}[h]
  \centering
  \begin{tikzpicture}
    \draw (0,0) -- node[near end, left]{$V_{1}$} ++(0,4) node[right]{$+\sigma$};
    \draw (2,0) -- node[near end, right]{$V_{2}$} ++(0,4) node[right]{$-\sigma$};
    \draw[<->, >=latex] (0.2,0) -- node[midway, below]{$e$} ++(1.6,0);
    \draw[->, >=latex] (-1,1) node[above left]{$S$} to[bend right=20] (-0.1, 0.5);
  \end{tikzpicture}
\end{figure}
\begin{cadre}{Capacité}
  La capacité $C$ est telle que:
  \[Q = C\cdot\pof{V1 - V2}\]
  \[C = \dfrac{\varepsilon_{0}S}{e}\]
\end{cadre}

\begin{cadre}{Densité volumique de courant}
  \[\vv \jmath = nq\vv v = \rho_{m}\vv v\]
  Dans les métaux:
  \[\vv \jmath = -ne\vv v\]
  \[i = \iint_{(S)}\vv\jmath\cdot\vv{dS}\]
\end{cadre}

\begin{cadre}{Conservation de la charge}
  \[\diverg \vv\jmath + \dfrac{\partial\rho}{\partial t} = 0\]
  La démonstration de ce résultat est à connaître, en voici les grandes lignes:
  \[-\dfrac{dq}{dt} = \ooint_{(S)}\vv \jmath \cdot\vv{dS}\]
  \[q(t) = \iiint_{(V)} \rho(M, t)d\tau\]
  Ensuite, dériver $q$, permuter la dérivation et la sommation, appliquer Ostrogradski. L'égalité des intégrales entraîne l'égalité des grandeurs sommées.
\end{cadre}

\begin{cadre}{Loi d'Ohm locale}
  \[\vv \jmath = \sigma\vv E\]
  Avec $\sigma$ la \emph{conductivité} du milieu, aussi notée $\gamma$.
\end{cadre}

\begin{cadre}{Loi de Joule locale}
  \[p = \vv\jmath\cdot\vv E\]
\end{cadre}

\begin{cadre}{Résistance d'un conducteur cylindrique}
  \[R = \dfrac{l}{\sigma S}\]
  Cette expression se retrouve rapidement avec un raisonnement purement intuitif du type \og Plus le fil est long plus la résistance est grande\fg{}.
\end{cadre}

\section{Magnétostatique}
\begin{cadre}{Force de Lorentz}
  \[ \vv f = q\pof{\vv E + \vv v \wedge \vv B}\]
\end{cadre}

\begin{cadre}{$\vv B$ est à flux conservatif}
  \[\ooint_{(S)}\vv B \cdot\vv{dS} = 0\]
  Or d'après le théorème d'Ostrogradski,
  \[\ooint_{(S)}\vv B \cdot\vv{dS} = \iiint_{(V)}\diverg \vv B d\tau = 0\]
  D'où:
  \[\diverg \vv B = 0\]
\end{cadre}

\begin{cadre}{Théorème d'Ampère}
  \[\oint_{(C)} \vv B(M)\cdot \vv{dl} = \mu_{0}I_{\text{int}}\]
\end{cadre}

\begin{cadre}{Discontinuité}
  La discontinuité d'un champ de part et d'autre d'une surface est en $\mu_{0}\vv{\jmath_{s}}$. Cela est à raprocher du $\frac{\sigma}{\varepsilon_{0}}$ en électrostatique.
\end{cadre}

\begin{cadre}{Champ dans un solénoïde}
  \[B_{\text{int}}=\mu_{0}nI\cdot\uz\]
\end{cadre}

\section{Dipôle magnétique}
\begin{cadre}{Moment dipolaire magnétique}
  \[\vv{\mathcal M} = IS\vv n\]
\end{cadre}

\begin{cadre}{Champ créé par un dipôle}
  \[\vv B(M) = \dfrac{\mu_0\mathcal M}{4\pi r^3}\pof{2\cos(\theta)\ur + \sin(\theta)\uth}\]
  Cette expression est totalement analogue à celle obtenue pour le dipôle électrostatique, il suffit en effet de remplacer $p$ par $\mathcal M$ et $\frac{1}{\varepsilon_{0}}$ par $\mu_{0}$.
\end{cadre}

\begin{cadre}{Moment des forces de Laplace}
\[\vv\Gamma(M) = \vv{\mathcal M}\wedge \vv B(M)\]
\end{cadre}

\begin{cadre}{Énergie Potentielle}
  \[E_{p} = -\vv{\mathcal M}\cdot \vv B\]
\end{cadre}

\begin{cadre}{Flux du champ magnétique}
  \[\Phi = \iint_{(S)}\vv B\cdot\vv{dS}\]
  \[\Phi = L_{1}i_{1} (+ M_{12} i_{2})\]
  Avec $L$ les coefficients d'\emph{auto-induction} et $M$ les coefficients d'\emph{iduction mutelle}.
  $L$ est bien sûr positif.
  On peut montrer que $M_{ij} = M_{ji}$.
\end{cadre}

\begin{cadre}{Inductance propre}
  Lorsque que l'on connait $\vv B$, il est possible de calculer le flux du champ, et donc l'inductance propre du circuit avec $\Phi =\Lambda i$.
\end{cadre}

\begin{cadre}{Énergie magnétique}
  \[U_{m} = \inv{2}L_{1}i_{1}^{2}+\inv{2}L_{2}i_{2}^{2}+\inv{2}M_{12}i_{1}i_{2}\]
  \[U_{m} = \iiint_{\text{espace}}\dfrac{B^{2}}{2\mu_{0}}d\tau\]
\end{cadre}

\end{document}
