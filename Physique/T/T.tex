\documentclass[french, a4paper, 11pt, twocolumn]{article}
\usepackage[left=1cm, right=1cm]{geometry}
\usepackage{marsu}
\usepackage{euler}
\usepackage[nointegrals]{wasysym}

\newtcolorbox{cadre}[2][]
{
  enhanced,
  attach boxed title to top left={yshift=-3.4mm, xshift = -2.3mm},
  adjusted title=#2,
  colback=white, colframe=black,
  colbacktitle=white, coltitle=black, fonttitle=\bfseries,
  breakable, sharp corners,
  boxed title style={colback=white, sharp corners, colframe=white},
  boxrule = 0.5mm, drop fuzzy shadow
}

\newcommand{\drz}[1]{\Delta_r#1\degres}

\title{Thermochimie}
\author{Martin \textsc{Andrieux}}
\date{}

\begin{document}
\maketitle

\begin{cadre}{Corps purs et corps simple}
  Une phase contenant plusieurs constituants est un \emph{melange}. Si elle n'en contient qu'un, c'est un \emph{corps pur}.
  Le corps pur est \emph{simple} s'il n'est formé que d'une seule sorte d'atomes. Dans le cas inverse, il est dit \emph{composé}.
\end{cadre}

\begin{cadre}{Enthalpie de réaction}
  \[\Delta_{r}H(T, P, \xi) = \pof{\dfrac{\partial H}{\partial \xi}}_{T,P}\]
  \[\drz{H}(T) = \pof{\dfrac{\partial H\degres}{\partial \xi}}_{T}\]
  \[\drz{H}(T) \approx \Delta_{r}H(T,P,\xi)\]
  \[\Delta H = \int_{\xi_{1}}^{\xi_{2}}\Delta_{r}H(T,P,\xi)=\drz{H}\cdot\pof{\xi_{2}-\xi_{1}} \]
\end{cadre}

\begin{cadre}{Loi de Hess}
  Si une équation est combinaison linéaire de plusieures autres, les $\drz{X}$ peuvent être calculés par combinaisons linéaires des $\drz{X}$ associés aux autres réactions.
\end{cadre}

\begin{cadre}{Chaleur reçue}
  À température constante:
  \[Q = \Delta H = \drz{H}(T)\cdot\pof{\xi_{2}-\xi_{1}}\]
  Cas général:
  \[\drz{H}(T_{0})\cdot\pof{\xi_{2}-\xi_{1}}+\int_{T_{0}}^{T_{f}}C_{p}(T)dT = 0 \]
  La réaction est dite \emph{endothermique} si $\drz{H} > 0$ et \emph{exothermique} si $\drz{H} < 0$
\end{cadre}

\begin{cadre}{Enthalpie libre}
  \[G = U + PV - TS\]
\end{cadre}

\begin{cadre}{Identités thermodynamiques}
  Pour un système de \emph{composition constante}:
  \[dU = TdS - PdV\]
  D'où:
  \[dH = TdS + VdP\text{ et }dG = VdP - SdT\]
\end{cadre}

\begin{cadre}{Potentiel chimique}
  \[\mu_{i} = \pof{\dfrac{\partial G}{\partial n_{i}}}_{T,P,n_{j\neq i}}\]
\end{cadre}

\begin{cadre}{Identité d'Euler}
  \[G = \sum n_{i}\mu_{i}\]
\end{cadre}

\begin{cadre}{Système fermé à $T$ et $P$ constants}
  \[dG_{T,P} = \sum \mu_{i}dn_{i} \leq 0\]
  À l'équilibre:
  \[dG_{T,P} = 0\]
\end{cadre}

\begin{cadre}{Activité d'un constituant}
  \[\mu(T,P) = \mu\degres(T) + RT\ln\pof{a}\]
  Où $a$ est l'activité du constituant.
\end{cadre}

\begin{cadre}{Entropie de réaction}
  \[\Delta_{r}S(T, P, \xi) = \pof{\dfrac{\partial S}{\partial \xi}}_{T,P}\]
  \[\drz{S}(T) = \pof{\dfrac{\partial S\degres}{\partial \xi}}_{T}\]
  \[\drz{S}(T) \neq \Delta_{r}S(T,P,\xi)\]
\end{cadre}

\begin{cadre}{Calcul de $\drz{S}$}
  \[\drz{S}(T) = \sum \nu_{i}S_{m,i}^{\circ}(T)\]
  \[\drz{S}(T) \sim \Delta_{r}\nu_{g}\cdot \unit{200}{\joule\reciprocal\kelvin\reciprocal\mole}\]
  Où $\Delta_{r}\nu_{g}=\sum_{(gaz)}\nu_{i}$ est la variation du nombre de moles gazeuses dans l'équation de la réaction.
\end{cadre}

\begin{cadre}{Enthalpie libre de réaction}
  \[\Delta_{r}G(T, P, \xi) = \pof{\dfrac{\partial G}{\partial \xi}}_{T,P} = \sum \nu_{i}\mu_{i}(T,P,\xi)\]
  \[\drz{G}(T) = \pof{\dfrac{\partial G\degres}{\partial \xi}}_{T} = \sum \nu_{i}\mu_{i}^{\circ}(T)\]
  \[\drz{G}(T) \neq \Delta_{r}G(T,P,\xi)\]
  \[\drz{G}(T) = \drz{H} - T\drz{S}\]
  \[\Delta_{r}G(T, P, \xi) = \drz{G}(T) + RT\ln\pof{Q_{r}}\]
  À l'équilibre:
  \[\drz{G}(T) + RT\ln\pof{K\degres(T)} = 0\]
\end{cadre}

\begin{cadre}{Relation isobare de Van't Hoff}
  Nous savons que:
  \[\drz{G}(T) + RT\ln\pof{K\degres(T)} = 0\]
  D'où, après division par $RT$ et dérivation:
  \[\dfrac{d\pof{\ln K\degres(T)}}{dT} = \dfrac{\drz{H}}{RT^{2}}\]
  On en déduit qu'une augmentation de la température déplace l'équilibre dans le sens endothermique.
\end{cadre}

\end{document}
